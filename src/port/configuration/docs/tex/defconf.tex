%% -*- Mode: LaTeX -*-

% defconf.tex --

% $Id: defconf.tex,v 1.3 2001/04/22 18:38:21 marcoxa Exp $

%\documentclass[a4paper]{article}
\documentclass{article}

\usepackage{latexsym}
%\usepackage{epsfig}
\usepackage{alltt}
%\usepackage{html}

%\input{progr-macros}

%\addtolength{\textwidth}{4cm}
%\addtolength{\textheight}{1cm}

\addtolength{\textwidth}{5cm}
\addtolength{\oddsidemargin}{-2.5cm}
\addtolength{\evensidemargin}{-2.5cm}

\addtolength{\textheight}{4cm}
\addtolength{\topmargin}{-1cm}

\newcommand{\CL}{\textsc{Common Lisp}}
\newcommand{\CLOS}{\textsc{CLOS}}
\newcommand{\Java}{\textsc{Java}}

\newcommand{\DEFCONFIGURATION}{\texttt{DEFCONFIGURATION}}
\newcommand{\defconfiguration}{\texttt{defconfiguration}}

\newcommand{\checkcite}[1]{{\textbf{[Missing Citation: #1]}}}
\newcommand{\checkref}[1]{{\textbf{[Missing Reference: #1]}}}

\newcommand{\missingpart}[1]{{\ }\vspace{2mm}\\
{\textbf{[Still Missing: #1]}}\\
\vspace{2mm}}

\newcommand{\specialnote}[1]{{\ }\vspace{2mm}\\
\emph{Note: #1}\\
\vspace{2mm}}

\newcommand{\notimpl}{\ddag}


\newcommand{\cfr}{\textit{cf.}}

\newcommand{\marginnote}[1]{%
\marginpar{\begin{small}\begin{em}
\raggedright #1
\end{em}\end{small}}}

\newcommand{\code}[1]{\texttt{#1}}
\newcommand{\object}[1]{\texttt{\textit{#1}}} % 'object' of discourse.
\newcommand{\clobject}[1]{\texttt{\textit{#1}}} % A CL 'object'.
\newcommand{\clhsterm}[1]{\texttt{\textit{#1}}} % A CLHS 'term'.
\newcommand{\syntaxnt}[1]{\emph{<#1>}}

\newcommand{\tm}{$^\mathsf{tm}$}


\title{
\DEFCONFIGURATION{}\\
{\normalsize A configuration tool for \CL{} programming.}\\
(Draft 1)
}

\author{Marco Antoniotti}

%\date{}


%\includeonly{}

\begin{document}

% HTML color directive

% \bodytext{BGCOLOR="\#FFFFFF"
% 	  TEXT="\#000000"
% 	  LINK="\#0099ff"
% 	  % VLINK="\#00ff99"
% 	  ALINK="\#FF00FF"}

\maketitle

\section{Introduction}

\CL{} applications have been traditionally distributed with a variety
of \texttt{defsystem}\footnote{The \texttt{defsystem} issue is a sore
one within the \CL{} community.  A variety of incompatible versions of
the \code{defsystem} (or \texttt{defsys}) macro has been floating
around for years. Some of the major packages (e.g. \texttt{PCL})
provide their own version, and the vendors distribute proprietary
versions of this ``application''.} definitions, or, worse, with a set
of \emph{script-like} files (usually named \texttt{build.lisp},
\texttt{load-foo.lisp}, and so on).  Of course there is nothing wrong
with this situation, but it would be nicer to make life easier for the
\CL{} programmer, by providing some basic tools that have become
common within other programming communities, most notably the C/C++
one and its descendants.

\subsection{Download, Install?}.

For the purpose of this introduction, let's consider what happens when
a piece of software is acquired and copied to your mass storage
device. Let's consider four more ``mainstream'' cases than ``\CL{}
programming''
\begin{enumerate}
\item	MS~Windows
\item	Mac
\item	\Java{} applications
\item	UN*X applications
\end{enumerate}

Cases 1 and 2 are very similar.  You usually have a self-extracting
archive, and/or a \texttt{setup} program that launches something like
the \textsf{InstallShield} application\checkref{missing reference}.

Case 3 (\Java{}) starts to appear more and more similar to cases~1
and~2 (especially on the MS~Windows platform).

These cases are interesting, because, with a minimum of insight, they
obviously exploit the relative homogeneity of the underlying platform
and provide a simple way to configure an application in a more or
less ``natural'' way\footnote{Of course the picture may not be so
rosy, but from the point of view of the user, not the configurer's, it
is the way things look like.}. In other words, installation and
configuration of your software are relatively easy.

Case 4 is more interesting, mostly because the underlying platform is
much more heterogeneous, both in executable and library formats, and
in installation location conventions (i.e. \texttt{/opt/\ldots},
versus \texttt{/usr/local/\ldots}, etc.).

The C/C++ and UN*X community has solved this problem with tools like
\texttt{imake} and \texttt{automake}/\texttt{autoconf}.  Whenever a
piece of software is acquired, the person in charge of installation
usually has just to issue the following command sequence at the shell
prompt.
\begin{verbatim}
  configure
  make
  make install
\end{verbatim}
The various options to \code{configure} (most notably the
\code{--prefix} option) allow the user to tailor the installation and
the check for required and conflicting packages.

The \CL{} does not have anything similar to the tools provided in the
four cases just discussed.  This is both unfortunate and strange, both
because the standard leaves some things ``up to the implementation'',
and because \CL{} implementations do not vary as much among themselves
after all.

This document describes a tool, \defconfiguration{}, which aims to
provide a way to ease the set up of a \CL{} application.


\section{A Tool for Configuring \CL{} Applications}

There are a number of reasons to provide a tool capable of configuring
a \CL{} application. Let's discuss two small cases.

\paragraph{Setting up Logical Pathnames.} \CL{} provides \emph{logical
pathnames} \cite{pitman96:_common_lisp_hyper} as a rather
nice\footnote{Sometimes it may be a little cumbersome, but generally
the idea works nicely.}  way to set up a ``canonicalized'' location
for various pieces of an application.

Unfortunately, the standard provides only an \emph{implementation
dependent} function to load the logical pathname
translations\footnote{I.e. \code{LOAD-LOGICAL-PATHNAME-TRANSLATIONS}.}
necessary for an application. In alternative the user can \code{SETF}
the \code{LOGICAL-PATHNAME-TRANSLATION} for a given ``logical host''.

In other words, either the user must know about the conventions of the
underlying implementation regarding the location of the default
translations, or s/he must resort to writing a piece of code doing the
needed \code{SETF}.  Hence the proliferation of \texttt{build.lisp}
files.


\paragraph{Generating ``Images''.} It seems to be common practice to
generate a \CL{} \emph{image} with a given application loaded in order
to produce an easily startable executable on either a UN*X, a
MS~Windows, or a Mac platform.

The process of creating an image is currently implementation dependent
(e.g. by using the function \code{SAVE-IMAGE} from the
\code{EXTENSIONS} package in CMUCL).  Moreover it has some (rather
minor in the year 2000) drawbacks: it uses up more disk
space (\CL{} images are not particularly small), and it tends to make
the software development cycle a little longer (i.e. it is the opinion
of the writer that an image should be used only to \emph{freeze} and
application and to avoid their use during the development cycle).

In other words, the use of images impairs the modular development of
libraries and applications\footnote{A personal
opinion, but with some logic.}. It would be worthwhile to make the
process of image generation ``localized'' and well constrained.

% say why this may or may not be desirable and why the lack of a
% configure makes the process cumbersome.

\subsection{Going \emph{declarative}: the \defconfiguration{} utility}

The tasks of specifying system dependencies and logical pathname
translations are the key steps to be performed to configure a \CL{}
system/application.  The \defconfiguration{} utility tries to ease and
centralize this tasks and to put them under user control (e.g. the
actual location of the \clobject{} translations is kept in
\emph{implementation dependent} places and formats
\cite{pitman96:_common_lisp_hyper}).

 The \defconfiguration{} utility solves a number of issues (but not
all). Mostly it gives the developer a centralized place where to keep
diverse requirements for a given \CL{} system/application, indexed on
the basis of the underlying operating system and \CL{} implementation.

Moreover, the \defconfiguration{} utility is
\emph{declarative}. I.e. it does not specify \emph{how} a
configuration is to be ``built''. It only specifies what is its
structure with respect to the underlying platform (that is: the
combination of operating system, \CL{} implementation, and, possibly,
specific machine).  The \defconfiguration{} utility will be described
in the following. Passages marked with a ``\notimpl{}'' should be checked
against Appendix~\ref{appendix:unimplemented} for their implementation
status.

With respect to the \code{autoconf} utility, \defconfiguration{} is more
rigid.  \code{autoconf} allows you to use the complete power of a
\emph{shell} to write configuration scripts. The claim is that
\defconfiguration{} needs not be as flexible, since \CL{} is generally
more portable than the regular C/C++ (or other language) application.

\defconfiguration{} appears as a pretty standard \CL{} \code{def...}
macro. All of its supporting code and the macro itself are defined in
the \code{CONF} package\footnote{\code{CONF} is a nickname.  The real
package name is \code{CL.EXT.CONFIGURATION}.} Its generic syntax is

\begin{alltt}
(\textbf{defconfiguration} \emph{<system-name>} \emph{<slot-list>}
   &rest  \emph{<configuration-clauses>})
\end{alltt}

\noindent
Figure~\ref{fig:defconf-example} shows a very simple configuration
form.
\begin{figure}
\begin{verbatim}
(defconfiguration "CONFIGURATION" ()
  (:library-location "/projects/lang/cl/CONFIGURATION/"
		     :os-type :unix)
  (:library-location "d:\\langs\\cl\\configuration\\"
		     :os-type :ms-windows)
  (:logical-pathname-host "CONF")
  (:required-package "cl.ext.properties")
  (:required-system "ENVIRONMENT")
  (:required-module a-standard-module
		    :pathnames-components ("ASM:f1" "ASM:f2" "ASM:d;f3"))

  (:special-translations
   :host "ASM"
   ("*.*.*" "/projects/lang/cl/modules/a-standard-module/"
    :os-type :unix)
   ("*.*.*" "/projects/lang/cl/modules/a-standard-module/"
    :os-type :ms-windows))
     
  (:special-translations
   ("CONF-PKG.*.*" "CONF-pkg.*.*")

   ("impl-dependent;cmucl-motif;*.*.*"
    "impl-dependent/cmucl-motif/*.*.*"
    :prefix "/projects/lang/cl/CONFIGURATION/special-stuff/"
    :prefix-configuration-key :cmucl-motif-location
    :os-type :unix)

   ("UTILITIES;*.*.*"
    "utilities/*.*.*"
    :os-type :unix))

  (:finally (configure-format *standard-output*
			      "done configuring CONFIGURATION.")))

\end{verbatim}
\caption{A simple \defconfiguration{} form.}
\label{fig:defconf-example}
\end{figure}

\noindent
The evaluation of a \defconfiguration{} form, results in the
definition of a \code{conf:setup} method which is specialized on the
configuration name\footnote{In reality, the evaluation also generates
an instance of the the \code{conf:configuration-template} class which
is ``registered'' under the \code{\emph{<system-name>}} key.
This machinery is in place for future development, in order to allow
the generation of different configurations for various platforms.}.
The guaranteed signature is

\begin{alltt}
(defgeneric setup (system-name &rest keys
                               &key
                               (logical-pathname-host "")
                               (library-location "")
                               (source-location "")
                               &allow-other-keys))
\end{alltt}

\noindent
The \code{system-name} parameter can be either a \clobject{symbol} or
a \clobject{string}.  The \defconfiguration{} macro may set up other
keyword arguments, and it may change the default of the
\code{logical-pathname-host}, \code{library-location}, and
\code{source-location}.  These changes depend on the structure of the
configuration clauses.

When executed, the \code{conf:setup} function will set up the computed
logical pathname translations and will ensure the presence of the
required packages, modules and systems in the current \CL{} session.

The configuration clauses are roughly divided in two logical groups:
clauses which deal with logical pathname setup, and clauses which deal
with ``other library'' dependencies.  Each clause has the form

\begin{alltt}
(\emph{<clause-keyword>} &rest \emph{<clause-arguments>})
\end{alltt}

\begin{itemize}
\item	\emph{logical pathname setup clauses}\\
	these clauses specify the logical pathname configuration for
	the system. They have the following
	\code{\emph{<clause-keyword>}}s (see Section~\ref{sect:dictionary} for
	a detailed description of each clause form).
	\begin{itemize}
	\item	\code{:logical-pathname-host}: specifies the overall
		system logical host (if not present, the
		\code{\emph{<system-name>}} is used after being
		stringified and capitalized).

	\item	\code{:source-location}: specifies the location of the
		source code for the system (a \clobject{string} or a
		\clobject{pathname} \notimpl{}).

	\item	\code{:library-location}: specifies the location of the
		library code for the system (a \clobject{string} or a
		\clobject{pathname} \notimpl{}). I.e. this is akin to
		the ``binary'' compilation results will be held.

	\item	\code{:special-translations}: these clauses specify
		other translations which may be needed for the
		correct functioning of the system.  Their syntax is
		more elaborated and it will be explained elsewhere in
		this document.
	\end{itemize}

\item	\emph{dependencies clauses}\\
	these clauses specify various requirements that must be met in
	order to have a ``good'' environment in which compile or run
	the application/system being configured.  The aim here is to
	take into account all the possible (read: known) ways in which
	a \CL{} system may be built and run.  The clauses are listed
	hereafter and they are ``dependent'' on the definitions set up
	by the logical pathname setup clauses.
	\begin{itemize}
	\item	\code{:required-package}\\
		this clause simply states that a given package (whose
		name is the sole argument in this clause) must be
		available in the current \CL{} session.  Otherwise an
		error is signaled.

	\item	\code{:required-module}\\
		this clause gets translated into a call to
		\code{REQUIRE}, which is executed by the
		\code{conf:setup} method.

	\item	\code{:required-system}\\
		this clause gets translated into a \emph{load
		operation} for the specified \clobject{system}, as
		defined by one of the many \code{DEFSYSTEM} or
		\code{DEFSYS} utilities available in the underlying
		\CL{} session.
	\end{itemize}

\end{itemize}

The code resulting from the processing of the clauses is executed in
the following order: logical pathname setup code, required module
forms, required system forms, required packages forms.  There is also
a \code{:finally} clause which contains an arbitrary form that can be
executed at the end of the execution of the \code{conf:setup} method.

\subsection{Extending \DEFCONFIGURATION{}}

New clause \emph{kinds} can be defined for \DEFCONFIGURATION{}.  The
system provides two generic functions (\code{CONF:PARSE-CONF-CLAUSE}}
and \code{CONF:BUILD-CONF-CLAUSE-CODE}) which can be used to extend
the behavior of \DEFCONFIGURATION{}.  The extra clause will be
processed last (w.r.t. the ``standard'' ones) and a the special clause
key \code{:CONFIGURATION-KEY} will be treated in a special way.

\paragraph{Implementation Note:} \emph{Please check the code for an
understanding of this machinery, which is now implemented in a very rough way}.

\section{\defconfiguration{} Dictionary}
\label{sect:dictionary}

\subsection*{Package \code{CL.EXT.CONFIGURATION}}

\subsubsection*{Uses:}

\code{COMMON-LISP}.

\subsubsection*{Nicknames:}

\code{CONF}.

\subsubsection*{Exports:}

\code{defconfiguration}, \code{setup}, \code{configure-format}.

\subsubsection*{Description:}

The package \code{CL.EXT.CONFIGURATION} contains all the symbols
related to the definition and the functioning of the
\defconfiguration{} macro, and of the \code{conf:setup} generic
function.

\subsection*{Macro \DEFCONFIGURATION{}}

\subsubsection*{Syntax:}

\textbf{defconfiguration} \emph{system-name} \emph{slots}
[[\emph{configuration-clause}]] $\Rightarrow$ \emph{system-name}

\begin{alltt}
slots ::= \emph{<defclass slots>}
configuration-clause ::= (:logical-pathname-host \emph{host}) |
                         (:source-location \emph{location}
                                           [[location-option]]) |
                         (:library-location \emph{location}
                                            [[location-option]]) |
                         (:special-translations [:host \emph{host}]
                                                [[special-translation]]) |
                         (:required-package \emph{package-name}) |
                         (:required-module \emph{module-name}
                                           [:pathname-components
                                            \emph{pathname-list}]) |
                         (:required-system \emph{required-sys-name}
                                           [:system-file-namestring
                                            \emph{namestring}]
                                           [:system-location
                                            \emph{logical-namestring}]
                                           [:system-type
                                            \emph{system-type-tag}]) |
                         (:documentation \emph{string}) |
                         (:finally \emph{form}+)
location-option ::= :os-type \emph{operating-system-designator}
special-translation ::= (\emph{logical-pathname-translation}
                         \emph{translation}
                         [[special-translation-option]])
special-translation-option ::= location-option |
                               :prefix \emph{translation} |
                               :prefix-configuration-key \emph{keyword}
\end{alltt}


\subsubsection*{Arguments and Values:}

\begin{description}
\item[{\emph{system-name}}] --- a \clhsterm{string designator}.
\item[{\emph{defclass slots}}] --- a \clhsterm{list} of slot
	definitions as in \clobject{defclass}.
\item[{\emph{host}}] --- a \clhsterm{logical host designator}.
\item[{\emph{location}}] --- a \clhsterm{string}
	% or a
	%\clhsterm{pathname designator}
	 to be interpreted as a
	\emph{to-wildcard} in a logical pathname translation.
\item[{\emph{logical-pathname-translation}}] --- a
	\emph{from-wildcard} part of a logical pathname translation
	(\cfr{} \code{(SETF LOGICAL-PATHNAME-TRANSLATIONS)}).
\item[{\emph{translation}}] --- anything coercible to a \clhsterm{pathname}.
\item[{\emph{prefix}}] --- anything coercible to a \clhsterm{pathname}
	with \emph{name} either \code{nil} or \code{:UNSPECIFIC} (i.e
	a pathname denoting a \emph{directory}).
\item[{\emph{operating-system-designator}}] --- a \clhsterm{symbol} (a
	\clhsterm{keyword}).
\item[{\emph{module-name}}] --- a \clhsterm{string designator}.
\item[{\emph{pathname-list}}] --- a list of \clhsterm{pathname designator}s.
\item[{\emph{package-name}}] --- a \clhsterm{package designator}.
\item[{\emph{system-type-tag}}] --- a \clhsterm{keyword} indicating a
	specific \code{DEFSYSTEM} or \code{DEFSYS} implementation.
\end{description}


\subsubsection*{Description:}

\defconfiguration{} processes each \emph{configuration-clause} and
builds a method for the \code{conf:setup} generic function.  The
method discriminates on \emph{system-name}.

Each clause admitting the \code{:os-type} option is processed by first
checking the associated \emph{operating-system-designator}. If
present, then its value is compared against the value of the
\code{cl.env:*operating-system*} variable. If not ``compatible''
(e.g. the clause contains \code{:os-type :mac-os}, and the value of
\code{cl.env:*operating-system*} is a subclass of \code{cl.env:unix}),
then the clause is \emph{discarded}, otherwise it is further processed. If
not present, then the \emph{operating-system-designator} is assumed to
be ``compatible'' with \code{cl.env:*operating-system*}.

The clauses which do not admit the \code{:os-type} option are
\emph{not discarded}.

If a clause is not discarded, then it is processed as follows,
according to its kind.
\begin{description}
\item	\code{:logical-pathname-host}\\
	the \clhsterm{host} is used as the ``main'' logical pathname
	host for the configuration.

\item	\code{:library-location}\\
	if the \emph{location} specified in this clause is not the
	empty string (\code{""}), it must denote a
	directory in the underlying file system. The \emph{location}
	is augmented with a wildcard \code{"**"} and used to set up
	the \code{system-name:**;*.*.*} translation. If the
	\emph{location} is the empty string or if no
	\code{:library-location} can be found, then the translation
	will be set using the ``current working directory'' (See
	``Notes:'' section).

\item	\code{:source-location}\\
	if this clause is supplied and the \emph{location} specified
	in this clause is not the 
	empty string (\code{""}), it must denote a
	directory in the underlying file system. The \emph{location}
	is augmented with a wildcard \code{"**"} and used to set up
	the \code{system-name:source;**;*.*.*} translation. Otherwise,
	the translation is set up using the
	\code{:library-location} clause. I.e. the
	\code{system-name:source;**;*.*.*} translation will be set to
	be the same of the ``library'' translation.

\item	\code{:special-translations}\\
	if present, this clause contains some extra translations which
	must be set up for the benefit of all the the
	\code{:required-}xxx clauses.
	It may contain a \code{:host} option. If so, then the
	associated \emph{host} is used to set up a new logical host
	with the translations specified in the list of
	\emph{special-translation}s.  If not present, the list of
	\emph{special-translation}s is added to the list of
	translations of the \emph{host} specified for the
	\code{:logical-pathname-host} clause.

	Each \emph{special-translation} is used as follows. The
	\emph{logical-pathname-translation} and the \emph{translation}
	are used as \emph{from-wildcard} and \emph{to-wildcard} when
	setting up the translation for the \emph{host} (which has been
	determined as per the description in the previous paragraph).
	However, the \code{:prefix} and the
	\code{:prefix-configuration-key} options change the way the
	\emph{translation} in transformed into the \emph{to-wildcard}.

	If the \code{:prefix-configuration-key} option is present,
	then the \emph{translation} is prepended with the value of the
	\emph{keyword} argument, which is added to the
	\code{conf:setup} argument list.  The default of such
	\emph{keyword} argument is either the empty string (\code{""})
	\notimpl, or the value of the \code{:prefix} option.
	If the \code{:prefix-configuration-key} option is not present,
	the keyword argument for \code{conf:setup} is not generated.

	If the the \code{:prefix} is present, and the
	\code{:prefix-configuration-key} option is not present, then
	the value associated to it is prepended to the
	\emph{translation} according to the following rules:
	\begin{itemize}
	\item	the value is \code{:source-location}\\
		the value prepended is the one associated with the
		\code{:source-location} clause.
	\item	the value is \code{:library-location}\\
		the value prepended is the one associated with the
		\code{:library-location} clause.
	\item	the value associated is a \clhsterm{pathname
		designator} denoting a directory\\
		the value is prepended to the directory component of
		\emph{translation}. (This operation may be rather involved).
	\end{itemize}

\item	\code{:required-package}\\
	the package specified is looked for with \code{FIND-PACKAGE}.
	If not found an error is signaled.

\item	\code{:required-module}\\
	the clause is transformed into a \code{REQUIRE} form which is
	executed in the context of the logical pathname translations
	set up by the translation clauses.  The full form is
	\code{(apply \#'require \emph{module-name} \emph{pathname-list})}.
	
\item	\code{:required-system}\\
	this clause checks whether a ``system'' (as defined by any of
	the ``known'' \code{DEFSYSTEM} or \code{DEFSYS}
	implementation) named \emph{required-system-name} is
	available on the computing platform.

	The ``availability test'' is performed based on the value of
	the \code{:system-type} option \notimpl{}.  This option can
	assume any of the following values\footnote{The list is
	currently incomplete.}: \code{:mk} (also \code{:make} and
	\code{:mk-defsystem}), \code{:allegro}, \code{:lispworks},
	\code{:lcl}, \code{:genera}, \code{:pcl}, and \code{:mcl}.
	Where \code{:mk} (\code{:make}, or \code{:mk-defsystem})
	denotes \code{MK:DEFSYSTEM}, \code{:pcl}
	denotes the \code{DEFSYS} implementation of \code{PCL}, and
	the remaining ones denote the other implementation dependent
	\code{DEFSYSTEM}.

	Each \code{defsystem} is assumed to provide a \emph{find} and
	a \emph{load} operation.  The \emph{find} operation is assumed
	to return a \clobject{system object} or \code{NIL}.
	The \emph{load} operation is assumed to be used to load a
	``defined'' system.

	The code resulting from the processing of this clause will
	perform the following algorithm.  If the system is not
	``found'' then the ``system file'' (i.e. the file presumably
	containing the \emph{defsystem} macro call) is loaded using
	the pathname resulting from the concatenation of the values of
	the \code{:system-location} option and of the
	\code{:system-file-namestring} option (which default to the
	\emph{logical-pathname-host} translation, and to the the
	\emph{system-name}\code{.system}, respectively). Of course, an
	error may be signaled by \code{LOAD} in this case.

	If a system definition has been loaded (hence the system can be
	``found'') then the appropriate \emph{load} operation is
	executed, trying to make the system effectively present in the
	\CL{} session\footnote{This sort of operation may also be
	performed by many \code{defsystem} utilities.  It is an
	intended -- and limited -- overlap in functionality.}.
\end{description}

\subsubsection*{Examples:}

See Figure~\ref{fig:defconf-example}.

\subsubsection*{Affected By:}

None.

\subsubsection*{Exceptional Situations:}

Many... All the underlying machinery for modules, packages, and
``systems'' does affect the behavior of \defconfiguration{}.

\subsubsection*{See Also:}

Logical Pathnames. \code{conf:setup}. The package
\code{CL.ENVIRONMENT} (\textbf{missing reference}).

\subsubsection*{Notes:}

The \defconfiguration{} macro uses the \code{CL.ENVIRONMENT} package to
abstract form the underlying computing platform and reliably define
notions like ``current working directory'', which are left in an
``implementation dependent'' state in the CLHS
\cite{pitman96:_common_lisp_hyper}.  (See
Appendix~\ref{appendix:unimplemented} for a list of unimplemented
features.)



%\subsection*{Generic Function \code{SETUP}}

\section{Conclusion}

Setting up a \CL{} application has not been a ``standardized'' (for an
appropriate definition of the term) task.  The tasks of specifying
system dependencies and logical pathname translations are the key
steps to be performed to configure a \CL{} system/application.  The
\defconfiguration{} utility tries to ease and centralize this tasks
and to put them under user control.

It is the author's hope that this code will be useful for any person
who wants to write \CL{} applications and systems.


\subsection{License Information}

The code is released under the LGPL.  Please refer to the GNU Web site
for more information -- \texttt{http://www.gnu.org}.



\bibliographystyle{plain}
\bibliography{General,Programming,Reactive}

\appendix

\section{Unimplemented features list}
\label{appendix:unimplemented}

There are many unimplemented features.  Here is a partial list.
\begin{enumerate}
\item	Clause discrimination checks only the operating system
	information; the \CL{} implementation information is
	effectively ignored.
\item	Running the \code{conf:setup} method is a one-shot
	operation. The system could be setup in such a way to interact
	with the user in order to fill in missing pieces of
	information.
\item	The \code{:source-location} and \code{:library-location}
	clauses (as well as the corresponding keyword arguments to the
	\code{conf:setup} method) cannot be \clobject{pathname}s
	yet. They should be.
\item	The \code{:source-location} clause is partially ignored in the
	current code setup.
\item	The defaulting behavior for the
	\code{:prefix-configuration-key} generated keyword argument
	(the empty string \code{""}) is to be implemented.
\item	The treatment of the \code{:system-type} option in the
	\code{:required-system} clause is far from complete. In
	particular a \CLOS{} based system to discriminate among the
	various \code{FIND-SYSTEM} could be easily set up.
\item	If the system in a \code{:required-system} clause is is not
	``found'' no error is signalled yet.
\end{enumerate}

\end{document}

% end of file -- defconf.tex --
